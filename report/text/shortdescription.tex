YAPL is a simple imperative programming language. The language is quite practical for smaller programs, but has also something to offer for the more skilled programmer.


YAPL is the abbreviation for Yet Another Programming Language. The language is an expression language, almost all constructs are expressions.

The language grammar is defined in an ANTLR4 grammar. The language is LL(1). The grammar can be found in Appendix \ref{app:antlr4} For extra proof that the language is LL(1), an ANTLR3 specification of the language is included with the LL(1) limit in Appendix \ref{app:antlr3}. 

The language supports the following language constructs:

\begin{itemize}
\item declaration: Variables and constants can be declared
\begin{itemize}
\item constants: defined with a default value that does not change
\item variables: declaration of variable name with its type
\end{itemize}
\item assignment: an expression which assigns a value to a variable and returns the result
\item compound expressions: expressions that open a local scope in which multiple declarations and expressions can be defined. A compound expression returns the last expression in the compound block.
\item conditional expression: based on a condition, either a true expression or false expression is evaluated.
\item while expression: as long as a predicate is true, an expression is evaluated.
\item binary expressions: expressions that operate a binary function on two expressions.
\item I/O functions: functions to read and write to standard in and out.
\end{itemize}