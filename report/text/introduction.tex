The goal of this project is to create a compiler for our own programming language. The compiler is a multipass compiler: The Abstract Syntax Tree(AST) is traversed multiple times to compile a program.

Yet Another Programming Language, later on referred to as YAPL, is created with the purpose to gain a better understanding of translators and programming languages. 

In this report we will explain in detail how YAPL is built and works. The compiler uses multiple passes to create the machine code. The input program first goes through a lexer that converts the character stream of the program to a token stream. This token stream is then given to a parser that converts the token stream to an AST. This AST is then visited by a context checker to verify if the contextual constraints of the language are met. Finally the AST is visited by a code generator that generates machine code by traversing the AST.
The lexer and parser are generated by ANTLR4. The code generator generates JVM assembly that is then assembled to JVM bytecode.

Finally, there are some tests. The tests use all the functionality of YAPL and tests correct and incorrect programs. For the correct programs the tests verify if these programs are correctly compiled and compile to correct and efficient byte code. The incorrect programs verify that the compiler reports human readable errors.